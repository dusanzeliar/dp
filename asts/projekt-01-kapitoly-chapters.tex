%===============================================================================
% Autoři: Michal Bidlo, Bohuslav Křena, Jaroslav Dytrych, Petr Veigend a Adam Herout 2018
\chapter{Úvod}

Tento text slouží jako ukázkový obsah šablony a současně rekapituluje nejdůležitější informace z předpisů a poskytuje další užitečné informace, které budete potřebovat pro tvorbu technické zprávy ke svojí práci. Než se šablonou budete dále pracovat, je třeba vědět, jak ji správně použít. To je stručně uvedeno v~příloze \ref{jak}.

I když některým studentům pro napsání dobré diplomové práce (bakalářská práce je také diplomová -- dostává se za ni diplom) stačí znát a dodržovat oficiální formální požadavky uvedené ve směrnicích a typografické zásady, často je výhodné před započetím psaní zjistit, jaké jsou osvědčené postupy pro psaní odborného textu a jak si práci usnadnit. Někteří vedoucí svým studentům připravili popisy osvědčených postupů, které vedly k desítkám úspěšně obhájených prací. Výběr nejzajímavějších postupů, které měli autoři této šablony k~dispozici ve chvíli její tvorby, je v níže uvedených kapitolách. Má-li Váš vedoucí svoji stránku s doporučenými postupy, tyto kapitoly můžete vynechat a řídit se pokyny svého vedoucího. Pokud takovou stránku nemá, může být přečtení níže uvedeného textu vhodnou přípravou na konzultaci o plánované struktuře a náplni textu práce.

Diplomová práce je rozsáhlé dílo a tomu odpovídá i technická zpráva. Ne každý je schopen si sednout a jednoduše ji napsat. Je třeba vědět, kde začít a jak postupovat. Jedním z možných přístupů je začít psaním klíčových slov a abstraktu, abyste si ujasnili, co je v~práci nejdůležitější. O tom pojednává kapitola \ref{abstrakt}.

Po sepsání abstraktu se lze pustit do psaní samotného textu technické zprávy. Typicky si nejprve připravíme základní strukturu práce, kterou pak budeme plnit textem. Kapitola \ref{struktura} se zabývá základními informacemi a radami pro psaní odborného textu, které Vám pomohou vyhnout se začátečnickým chybám, a stanovením nadpisů kapitol a přibližných rozsahů jednotlivých částí práce. V závěru kapitoly je pak uveden přístup, kterým si lze psaní technické zprávy značně usnadnit.

Diplomové práce v oblasti informačních technologií mají určitou typickou strukturu. Po~úvodu bude následovat kapitola či kapitoly zabývající se shrnutím současného stavu, který bude v následujících kapitolách zhodnocen a bude navrženo řešení, které bude implementováno a otestováno. V závěru pak budou výsledky vyhodnoceny a bude navržen budoucí vývoj. I když se názvy a rozsahy kapitol v různých pracích liší, vždy tam lze najít kapitoly odpovídající této struktuře. Kapitola \ref{kapitoly} se zabývá obsahy typických kapitol, které se v diplomových pracích z oblasti IT vyskytují. Většina studentů ve svojí práci pravděpodobně využije pouze určitou podmnožinu popsaných kapitol, která je pro jejich práci relevantní. Uvedené popisy a rady mohou pomoci jak s~rozhodnutím, zda danou kapitolu uvést, tak i~s~vnitřní strukturou a samotným obsahem kapitoly.

Za závěrečnou kapitolou práce vždy následuje seznam použité literatury. Citacemi, které tento seznam tvoří, a odkazy na ně se zabývá kapitola \ref{citace}. Byť to tak nezkušený student nemusí vnímat, je seznam použité literatury a odkazy na něj pro práci zcela zásadní. Hodnocení práce s literaturou a citací tvoří jednu z důležitých částí posudku oponenta a bude-li chybět jediná položka, může to vést k hodnocení stupněm F, následnému disciplinárnímu řízení za plagiátorství a k vyloučení z nedokončeného studia. Nesprávná práce se zdroji může mít i další důsledky -- v roce 2018 stála křesla dva členy české vlády. Proto prosím citacím věnujte odpovídající pozornost. 

Po dokončení textu je nutné zjistit, jaké požadavky jsou kladeny na vysokoškolskou kvalifikační práci na FIT VUT v~Brně, a dořešit případné nedostatky. Formální požadavky jsou uvedeny ve směrnicích a na webových stránkách, které jsou zmíněny v kapitole \ref{formality}. Tato kapitola obsahuje i požadované rozsahy jednotlivých typů prací a další vybrané informace z~předpisů a doporučení. V závěru kapitoly je uveden přehled nejčastějších chyb, se kterými se oponenti setkávají a kterým byste se měli vyhnout. Hodnocení formální úpravy práce je pak další z důležitých součástí posudku oponenta.

Po odstranění formálních nedostatků lze práci odevzdat. Před odevzdáním práce si můžete projít kontrolní seznam (tzv. \uv{checklist}) uvedený v příloze \ref{checklist}. Samotné odevzdání listinné i elektronické verze práce je pak popsáno v kapitole \ref{odevzdani}.

V závěrečné kapitole \ref{zaver} je pak uvedeno shrnutí toho, co se lze přečtením tohoto textu dozvědět, a to nejdůležitější, na co je třeba myslet před odevzdáním práce.


\chapter{Testovanie softwaru}
\label{testing}
Testovanie softvéru je súbor procesov analyzujúcich softvér za účelom vyhodnotenia jeho vlastností a detekcie rozdieľov medzi aktuálnym a požadovaným stavom \cite{Standard}. Táto kapitola najskôr predstavuje základné úrovne a druhy testovania. Následne popisuje testovanie založené na dátach a s tým spojenú rozhodovaciu tabuľku.



 \cite{Ist}\cite{Gst}\cite{Ast}\cite{Tgp}\cite{Dig}\cite{Testos}\cite{Standard}.
\bigskip
\section{Úrovne testovania}
Testy sú vytvárané na základe špecifikácií a požiadaviek, dizajnových artefaktov alebo zdrojového kódu. Rôzne úrovne testovania sprevádzajú rozdieľne vývojárske aktivity \cite{Ist}.
\subsection*{Jednotkové testovanie}
Jednotkové testovanie je zamerané na jednotky, predstavujúce najmenšie testovateľné komponenty testovaného systému. Zmyslom jednotkového testovania je validácia správania jednotky voči jej dizajnu. Zameriava sa na chyby na najnižšej úrovni. Testy vytvárajú samotný programátori počas vývoja.    
\subsection*{Integračné testovanie}
Integračné testovanie je cielené na korektnú komunikáciu medzi rozhraniami jednotiek, ktoré sú pre tento účel zlučované do podsystémov. Úlohou integračného testovania nie je nájdenie chýb v jednotlivých integrovaných jednotkách, ale overenie ich korektnej integrácií. Predpokladá sa, že tieto chyby boli eliminované pri jednotkovom testovaní. Odhaľuje chyby v rozhraniach a stavoch jednotiek. Pri väčšom množstve rozhraní je vhodné zvoliť špecifický prístup k integračným testom. Obvyklé stratégie sú zdola nahor, zhora dole, funkcionálna integrácia a veľký tresk \cite{Gst}. Obvykle ich tvoria vývojári alebo testeri v rámci tímu.
\subsection*{Systémové testovanie}
Systémové testovanie je zamerané na nájdenie chýb vo vlastnostiach plne integrovaného systému. Testovaný systém je tvorený komponentami, ktoré už úspešne prešli integračnými testami. Cieľom je detekcia nekonzistentností mezi týmito komponentami a systému ako ceľku voči špecifikácií požiadavkov. Systémové testovanie vykonáva oddelená skupina testerov mimo vývojový tím. 
\subsection*{Akceptačné testovanie}
Akceptačné testovanie je proces s účelom overenia softvéru voči počiatočným stanoveným požiadavkom zákazníka jeho aktuálnych potrebieb. Často sa na ich vytváraní podieľa expert na doménu, pre ktorý sa softvér vyvýja. Zvyčajne je vytvorený zákazníkom alebo koncovým užívateľom a overuje, že dané riešenie pre užívateľa funguje \cite{Ast}.  

\section{Dynamické testovanie}
Existuje mnoho prístupov k testovaniu softvéru. Na najvyššej úrovni sa testovanie rozdeľuje na {\it statické} a {\it dynamické} \cite{Ist}. Techniky, ktoré analyzujú a skúmajú program bez nutnotsti jeho spusteniam za účelom verifikácie spadajú do skupiny statického testovania. Zahňuje posudzovanie dokumentov, kódu a jeho statickú analýzu (základná statická analýza zvyčajne prebieha na úrovni kompilátorov). Druhá skupina techník spadá do skupiny dynamického testovania, ktorá je zameraná na analýzu dynamického správania kôdu s cieľom jeho validácie. Prerekvizitou je úspešná kompilácia a spustenie kódu. Zahrňuje prácu so softvérom, kedy pre špecifické vstupy overuje a analyzuje správnosť výstupov.

 Dynamické testovanie môže byť ďalej rozdelené na {\it funkcionárne} a {\it nefukncionárne}. Kým funkcionárne testovanie adresuje splňenie požiadaviek, nefukcionárne je mierené na ostatné oblasti ako bezpečnosť, výkonnosť, použiteľnosť, správa pamäti a iné. Podľa znalosti kôdu sa delí na {\it black-box, white-box a grey-box} testovanie.
 
  Táto práca sa zamerieva na testovanie založené na dátach, ktoré vychádza z funkcionálneho black-box testovania, ale výsledný nástroj môže byť prínosný aj pre iné prístupy.

       
\subsection*{Black-box testovanie}
{\it Black-box testovanie} (tiež známe ako {\it testovanie založené na dátach}) zoskupuje techniky tvorby testovacích prípadov na základe špecifikácií podľa analýzi popisu softvéru bez znalosti jeho vnútornej štruktúry  \cite{Ast}. Ich hlavným zameraním je odhalenie okoľností, pri ktorých sa systém správa odlišne od špecifikácií.  Testovacie dáta závisia na popise očakávaní od testovaného softvéru napríklad vo forme manuálu či popisu procesu.

Black-box testovanie môže byť využité na všetkých úrovniach testovania. Pre nižšie úrovne jednotkového a integračného testovania sa dá využiť ako počiatočný bod pre tvorbu testov na základe dizajnu alebo aj požiadaviek. Veľmi užitočné sú na vyžších úrovniach, teda systémová a akceptačná, kde sú testy založené na požiadavkách  \cite{Gst}.

\subsection*{White-box testovanie}
Techniky white-box testovania vytvárajú testovacie prípady podľa vnútornej štruktúry komponenty alebo systému. Hlavný dôraz kladú na vetvy, jednotlivé podmienky a výrazy tradične v zdrojovom kóde. Primárne sa využívajú v jednotkovom a integračnom testovaní. Všetky testovacie techniky tohoto druhu od testera vyžadujú znalosť danej štruktúry, teda programovacieho jazyka \cite{Gst}.    
\subsection*{Grey-box testovanie}
Mezi white-box a black-box testovaním je mnoho úrovní grey-box testovania ktoré predstavuje ich kombináciu. Testovacie prípady sú tvorené so znalosťou architektúry, algoritmov, vnútorných stavov alebo iného vysoko úrovňového popisu správania.
\section{Testovanie založené na dátach}
Jednoduché automatizované testovacie skripty obsahujú pevne dané testovacie dáta. Zmena týchto dát obvykle vyžaduje zmenu v zdrojovom kóde skriptu, čo môže viesť k viacerím komplikáciám. Ak je test neprehľadný, dlhý alebo neštrukturovaný, jednoduchá zmena v dátach je náročná aj pre skúsených expertov. Rovnako vzniká riziko zavedenia novej chyby. Pri tvorbe nových testov, líšiacich sa len v testovacích dátach, často dochádza k skopírovaniu kódu a následnej modifikácii dát. Pritom vzniká duplicita kódu a testovacie prípady sú ťažko udržateľné. 

Pri veľkých testovacích sadách sú pre spomenuté problémy skripty s pevne danými dátami len ťažko použiteľné. {\it Testovanie založené na dátach (Data-driven testing)} je metodológia , v ktorej sa opakovane vykonávajú rovnaké kroky skriptu s použitím externých dátových zdrojov. Takéto dáta musia byť ľahko editovateľné aj testerom bez znalosti zdrojového kódu. Umožňuje mu tým sústrediť sa len na tvorbu testovacích prípadov. Výhody metodológie sú zreteľné najmä pri aplikáciách s častými zmenami funkcionality. Hlavné výhody sú nasledovné \cite{Sttc}:
\begin{itemize}
	\item{Testy založené na dátach dosahujú vysoké pokrytie kódu testovacími prípadmi a zároveň minimalizujú množstvo kódu, ktoré je potrebné napísať a udržiavať}
	\item{Uľahčuje vytváranie a spúšťanie veľkého množstva testovacích podmienok}
	\item{Testovacie dáta môžu byť navrhnuté a vytvorené pred tým, ako je aplikácia pripravená na testovanie}
	\item{Rozhodovacie dátové tabuľky môžu byť použité pri manuálnom testovaní}
\end{itemize}
\begin{figure*}[h]\centering
	\centering
	\includegraphics[width=3.5in,height=2.2in]{obrazky-figures/Data-driven_testing.png}\\[1pt]
	\caption{Diagram záladnej štruktúry testovania založeného na dátach .}
	\label{Tdd_img}
\end{figure*}
\section{Editácia a uloženie testovacích dát}
Využívané testovacie dáta sa obecne skladajú z kombinácie vstupných a očakávaných výstupných dát. Daný typ dát sa najlepšie popisuje formou {\it rozhodovacích tabuliek}. Rozhodovacia tabuľka v najjednoduchšej forme poskytuje vstupy ako aj očakávané výstupy na jednom riadku. Pri tvorbe tabuľky je dôležitá správna identifikácia všetkých vstupných dát a ich rozdelenie do {\it domén}. Výber konkrétnych hodnôt záleží od zvoleného prístupu, najčastejšie na základe {\it analýzy hraničných hodnôt} a {\it rozdelenia na evivalenčné triedy}. 

Vzhľadom k charakteru dát sa k ich editácii prirodzene ponúka využitie tabuľkových procesorov (anglicky{\it spreadsheet}). Prácu s danými programmi obvykle zvládajú testeri ako aj ľudia z oblasti biznisu, čo uľahčuje ich rýchle zapojenie. Dané programy sa často používajú aj na jednoduchý manažment testov pre manuálne testovanie. V tomto prípade sa dáta môžu zdieľať s automatizovanými testami a predchádať tak ich zbytočnej redundancii. Príklad tabuľky je uvedený na obrázku \ref{Dt_img}. 
\begin{figure*}[h]\centering
	\centering
	\includegraphics[width=6.0in,height=2.2in]{obrazky-figures/decision_table.png}\\[1pt]
	\caption{Príklad rozhodovacej tabuľky pre uplatnenie zľavy vytvorenej v tabuľkovom editore.}
	\label{Dt_img}
\end{figure*}

Formáty uloženia tabuliek spadajú do viacerých kategórií. Prostý databázový soubor (anglicky flat file database) je jednoduchá databáza (väčšinou tabulka) uložená v textovom súbore ve forme prostého textu. Obvykle používané formáty sú napríklad hodnoty oddelené čiarkami (CSV), hodnoty oddelené tabulátormi (CSV, TXT) alebo iný špecifický formát (variácie XLS). Rovnako sú stále viac využívané štrukturované formáty ako XML a Json. Súčasné programovacie jazyky majú knižnice pre ich načítanie a spracovanie, čo výrazne uľahčuje ich využitie. Prosté dabázové súbory môžu mať problémy pri výraznom rozšírovaní. Rovnako je v nich neprehľadné uchovávanie rôznych konfigurácií a verzií. 

Pre veľké množstvo testovacích dát je vhodné využitie relačnej databázy. Umožňuje editáciu prostredníctvom skriptov ako aj pomocou grafických editorov.       
 
\section{Rozdelenie vstupných domén na ekvivalenčné triedy}
Zmyslom tvorby testovacích prípadov je nájdenie 
\section{Rozhodovacia tabuľka}
\section{Cause effect graph}
\section{Testovanie založené na dátach}
\section{Generovanie testovacích dát}
{TODO {\it Kritérium pokrytia} je pravidlo alebo predpis pre systematické generovanie požiadavkov na test. {\it Pokrytie} je miera, udávajúca ako dobre je dané krytérium
	
	
	Techniky black-box testovania majú asociované {\it kritéria pokytia}, kde aplikácia jednotlivých techník konkretizuje ich kritérium pokrytia z kritéria pokrytia požiadavkov na špecifické kritérium pre danú techniku.}  

Základné techniky black-box testovania sú nasledovné:
\begin{itemize}
	\item{Rozdelenie vstupných domén na ekvivalenčné triedy}
	\item{Cause-effect graf}
	\item{Rozhodovacie tabuľky}
	\item{Rozhodovacia tabuľka}
	\item{Provedeným výzkumem jsme zjistili, že.}
\end{itemize}


\noindent Za prvé – na abstraktu záleží. Za druhé – není těžké ho napsat. Pojďme na to.


\subsection*{K čemu je abstrakt}
Abstrakt slouží k \bf vyhledávání\rm, společně s názvem dané vědecké práce a seznamem klíčových slov. Tyto části (snad s výjimkou názvu) nejsou přímo součástí textu a nečeká se, že někdo, kdo by zasedl ke čtení dané vědecké práce, bude číst je. To, že práci čte, znamená, že už se dostal za fázi čtení abstraktu. Abstrakt mu slouží ve chvíli, kdy se ještě rozhoduje, \bf zda vůbec \rm text číst.

Když někdo tam venku hledá odpověď na svůj problém, zadá knihovnici nebo dnes spíše vyhledávacímu serveru klíčová slova, která se jeho potíží dotýkají. Na základě shody těchto klíčových slov a klíčových slov uvedených autory dostane seznam názvů prací, které by mu mohly nabízet řešení. Dobře sestavený název práce badateli pomůže vytipovat takové texty, které by mohly mít vztah k jeho problému a mohl by se zajímat o jejich přečtení.

A tady právě přichází na scénu abstrakt. Badatel si čte abstrakt vytipovaných prací a~rozhoduje se, zda práci skutečně chce číst, nebo jestli se v tomto případě jeho filtr založený pouze na jednořádkovém nadpisu zmýlil.

V tuto chvíli obvykle ještě nemá stažené nějaké PDF s celým textem, natož aby měl v ruce vytištěný fascikl. Abstrakty jsou určeny k tomu, aby byly \bf mimo text\rm , aby ležely na serverech agregujících vědecké texty. Proto první pravidlo je, že abstrakt musí fungovat samostatně -- pokud obsahuje odkazy do literatury nebo se odvolává na text (\uv{Výkonnost metody je přehledně shrnuta na straně 51.}), nedělá badateli dobrou službu, což badatel ocení tím, že si o autoru nepomyslí nic hezkého, práci si nepřečte a autora neocituje.

\subsection*{Kdy a jak psát Abstrakt}
Může dávat smysl psát abstrakt na závěr celého psaní -- jako shrnutí a skutečné anotování sepsaného díla. Já jsem vyznavačem opačného přístupu -- abstrakt píšu na samém začátku. Když píšu vědecký článek, začínám sepsáním velkého počtu klíčových slov, jež se textu dotýkají. Bývá jich více, než potom uvedu jako ona charakteristická klíčová slova používaná k indexování. Ujasňuji si tím prostor, kde se článek pohybuje -- o čem je třeba hovořit, co je v textu podstatné, čeho se dotýká. Hned po ujasnění klíčových slov formuluji nadpis a~právě abstrakt. 

Považuji za mimořádně užitečné ujasnit si právě ony čtyři části abstraktu -- Jaký problém se řeší? Jaké řešení práce nabízí? Jaké jsou přesně výsledky? Jaký je jejich význam? Když je toto jasné, text se píše skoro sám. Pokud toto má být nejasné, jak u všech všudy je možné vůbec dát dohromady smysluplnou větu v samém textu?

\subsection*{Doporučená struktura abstraktu}
Abstrakt vědecké práce se může skládat ze čtyř částí a pak být opravdu užitečný. Každá část se bude skládat z nějakých dvou, tří vět, někdy postačí jedna.

V byznysu se vžil slovesný útvar \uv{elevator pitch} -- představení ve výtahu. Ne náhodou jeho struktura připomíná právě doporučovanou strukturu abstraktu. Opravdu, autor odborného textu má do abstraktu napsat právě to, co by říkal o své práci, kdyby na to měl nejvýše dvě minuty a nemohl použít žádných slajdů, obrázků, textu. O čem by tedy měl mluvit?

\paragraph{První část -- Jaký se řeší problém? Jaké je téma? Jaký je cíl textu?}
\begin{itemize}
  \item{Tato práce řeší.}
  \item{Cílem této práce je.}
  \item{Zaměřil jsem se na.}
\end{itemize}
Nepatří sem úvodní pohádky charakteristické pro špatný odborný sloh: \uv{Naše poslední pětiletka staví před nás nové a smělé cíle}, \uv{S rozvojem výpočetní techniky a zejména zobrazovacích zařízení je stále důležitější \ldots} Ty nepatří do dobrého textu nikam, ale do~abstraktu tím méně. Pokud dokážete vyjádřit účel svého textu v jedné větě o pár slovech, udělejte to a nepřidávejte nic navíc. Stručnější zde vždy znamená lepší.

\paragraph{Druhá část -- Jak je problém vyřešen? Cíl naplněn?}
\begin{itemize}
  \item{Zvolený problém jsem vyřešil pomocí toho a toho.}
  \item{V řešení bylo použito metody té, postupu toho a analýzy oné.}
  \item{Práce představuje algoritmus takový, který.}
  \item{Data jsem zpracovával pomocí těch a těchto nástrojů a provedl vyhodnocení takové.}
  \item{Podstatou našeho algoritmu je.}
\end{itemize}

Pokud je podstatou sepisovaného odborného textu nová metodologie (= \uv{jak něco dělat}), patří přesně sem její popis. Pokud se představované řešení skládá ze tří částí, pravděpodobně v této části abstraktu budou tři věty, z nichž každá se bude věnovat jedné části řešení. Dobrý abstrakt v této části bude upřímný a přesný -- nebude si schovávat \uv{odhalování svých tajemství} až do textu. Vágní formulace podstaty řešení v abstraktu obvykle znamená, že autoři buď neumí psát a nebo vlastně nemají o čem -- ani jedno není zrovna výzva ke stažení a čtení mnoha stran textu.

\paragraph{Třetí část -- Jaké jsou konkrétní výsledky? Jak dobře je problém vyřešen?}
\begin{itemize}
  \item{Podařilo se dosáhnout úspěšnosti 87,3\,\%.}
  \item{V práci jsme vytvořili systém, který.}
  \item{Vytvořené řešení poskytuje ty a ty možnosti.}
  \item{Provedeným výzkumem jsme zjistili, že.}
\end{itemize}

Není špatným zvykem uvést v této části konkrétní číslo -- \uv{existující metodu XY jsme zrychlili pětkrát}. Pokud přínos práce není možné shrnout do dvou nebo tří vět, někde je něco velmi špatně a celý text pravděpodobně nestojí za psaní.

\paragraph{Čtvrtá část -- Takže co? Čím je to užitečné Vědě a čtenáři?}
\begin{itemize}
  \item{Přínosem této práce je.}
  \item{Hlavním zjištěním je.}
  \item{Hlavním výsledkem je.}
  \item{Na základě zjištěných údajů je možné.}
  \item{Výsledky této práce umožňují.}
\end{itemize}

Při psaní vědeckých článků já sám obvykle bojuji s podobností části třetí a čtvrté. Vskutku, obě hovoří o tom, co jsou výsledky a přínosy textu. Účelem třetí části je jmenovitě a konkrétně jmenovat dosažené výsledky, úkolem části čtvrté je interpretovat jejich význam. Asi ničemu nevadí, když tato dvě sdělení do jisté míry splynou a část třetí a čtvrtá nejen že nemají každá vlastní odstavec, ale prolínají se dokonce ve společných větách.

\paragraph{Nultá část -- O co jde? Kde jsme?}
\begin{itemize}
  \item{Práce je řešena v kontextu tom a tom.}
  \item{Nauka ta a ta se zabývá studiem toho a toho.}
  \item{Stavíme na těchto a oněch nedávných pokrocích v naší oblasti.}
\end{itemize}

Někdy je nutné na sám začátek abstraktu vložit kratičké uvedení kontextu, ve kterém se~celá záležitost vlastně odehrává. Může to být přínosné~u vskutku obskurního a esoterického oboru, který leží stranou hlavního proudu. Obvykle tato část ovšem nebývá nutná a~věty v~ní obsažené bývají prototypy ohavné, rádobyodborné vaty. Je dobrou praxí zapomenout, že se tato část v abstraktu může vyskytovat. Když někdo, kdo je odborníkem v~oboru práce, přece po přečtení abstraktu zavrtí hlavou: \uv{Vůbec nevím, o čem tady můžete psát,} pouze tehdy je vhodné vložit nějaké věty s uvedením kontextu.

\subsection*{Inovace není Ignorance}

Popisuji v tomto textu jakýsi obecný model obecné diplomky. Ještě ke všemu se na začátku zaklínám, že to je můj názor a vkus a jsem zvědavý na názory a vkusy alternativní (což jsem!). Každý diplomant (Mgr. i Bc.) přitom cítí, že jeho diplomka je speciální a výjimečná. Tudíž se nebude držet nějakého schématu, které slouží pro běžné a průměrné diplomky -- tj. pro ty ostatní. Setkávám se s dobrými důvody, proč se od výše naznačeného schématu odchýlit a každoročně některým studentům odchýlení od schématu sám doporučuji. Vskutku, každá diplomka je jedinečná a zvláštní. Kdyby ne, nemusely by se psát, stačilo by je kopírovat. Ovšem vždycky před tím, než vybočíte ze standardního a kanonického způsobu organizování odborného textu, dejte si tu práci ho poznat, pochopit a zvládnout. Způsob vědecké práce, strukturování odborného textu, nebo třeba citování pramenů, může vypadat rigidně a neohrabaně, je to ale zatím ten nejlepší způsob, který jsme jako lidstvo dokázali vymyslet. Pokud ho ovládnete, pochopíte jeho výhody a nevýhody a inovujete ho, je to v pořádku a jste vítáni. Pokud se jím odmítnete zabývat, pravděpodobně neprovedete hodnotnou inovaci, ale vytvoříte \uv{paskvil}.


\chapter{Stromové štruktúry} 
\label{struktura}

V této kapitole jsou nejprve uvedeny obecné zásady pro psaní odborného textu a po nich následuje detailnější popis doporučeného postupu přípravy struktury a základní osnovy práce.

Před začátkem psaní textu práce je vždy vhodné zeptat se svého vedoucího, co Vám poradí a zda nemá nějakou svoji aktuální stránku s radami a pokyny. Jeho zaměření bude pravděpodobně odpovídat zaměření Vaší práce a poradí Vám tu nejvhodnější strukturu, které byste se měli držet. Dozví-li se autoři tohoto souboru o další sbírce užitečných rad, jistě sem v budoucnu budou zařazeny.

Tento text se zaměřuje na obecná doporučení a obecnou strukturu práce, kterou je vždy potřeba  modifikovat a popřemýšlet o ní na základě konkrétního zadání \cite{Cernocky}.

\section{Grafová reprezentácia}
\section{Porovnávanie dát}
\section{Prehľad používaných formátov}
\section{Problém generovania}
\section{Existujúce generátory}

Následující pokyny jsou dostupné též na školních webových stránkách~\cite{fitWeb}. Přehled základů typografie a tvorby dokumentů s využitím systému \LaTeX{} je uveden v~knize od~Jiřího Rybičky~\cite{Rybicka}.

Hodnocenou součástí potenciálního inženýra je mimo jiné i jazyková kvalita a čistota. Naším cílem je vytvořit jasný a~srozumitelný text. Vyjadřujeme se proto přesně, píšeme dobrou češtinou či slovenštinou (případně angličtinou) a~dobrým slohem podle obecně přijatých zvyklostí. Předpokládá se dodržování pravopisných norem zvoleného jazyka práce a dodržování odborného názvosloví. Slangové výrazy jsou nepřípustné. Při pochybnostech o~překladu či přepisu cizích pojmů využijte literatury dostupné v knihovně FIT. 

Text má upravit čtenáři cestu k~rychlému pochopení problému, předvídat jeho obtíže a~předcházet jim. Dobrý sloh předpokládá bezvadnou gramatiku, správnou interpunkci a~vhodnou volbu slov. Snažíme se, aby náš text nepůsobil příliš jednotvárně používáním malého výběru slov a~tím, že některá zvlášť oblíbená slova používáme příliš často. Pokud používáme cizích slov, je samozřejmým předpokladem, že známe jejich přesný význam. Ale i~českých slov musíme používat ve správném smyslu. Např. platí jistá pravidla při používání slova {\it zřejmě}. Je {\it zřejmé} opravdu zřejmé? A~přesvědčili jsme se, zda to, co je {\it zřejmé} opravdu platí? Pozor bychom si měli dát i~na příliš časté používání zvratného se. Například obratu {\it dokázalo se, že \ldots{}} zásadně nepoužíváme.

Za pečlivý výběr stojí i~symbolika, kterou používáme ke {\it značení}. Máme tím na mysli volbu zkratek a~symbolů používaných například pro vyjádření typů součástek, pro označení hlavních činností programu, pro pojmenování ovládacích kláves na klávesnici, pro pojmenování proměnných v~matematických formulích a~podobně. Výstižné a~důsledné značení může čtenáři při četbě textu velmi pomoci. Je vhodné uvést seznam značení na začátku textu. Nejen ve značení, ale i~v~odkazech a~v~celkové tiskové úpravě je důležitá důslednost.

S tím souvisí i~pojem z~typografie nazývaný {\it vyznačování}. Zde máme na mysli způsob sazby textu pro jeho zvýraznění. Pro zvolené značení by měl být zvolen i~způsob vyznačování v~textu. Tak například klávesy mohou být umístěny do obdélníčku, identifikátory ze~zdrojového textu mohou být vypisovány {\tt písmem typu psací stroj} a~podobně.

Uvádíme-li některá fakta, neskrýváme jejich původ a~náš vztah k~nim. Když něco tvrdíme, vždycky výslovně uvedeme, co z~toho bylo dokázáno, co bude dokázáno v~našem textu a~co přebíráme z~literatury s~uvedením odkazu na příslušný zdroj. V~tomto směru nenecháváme čtenáře nikdy na pochybách, zda jde o~myšlenku naši nebo převzatou z~literatury.

Abychom mohli napsat odborný text jasně a~srozumitelně, musíme splnit několik základních předpokladů:
\begin{itemize}
\item Musíme mít co říci,
\item musíme vědět, komu to chceme říci,
\item musíme si dokonale promyslet obsah,
\item musíme psát strukturovaně. 
\end{itemize}

\subsection*{Musíme mít co říci}
Nejdůležitějším předpokladem dobrého odborného textu je myšlenka. Je-li myšlenka dost závažná, tak přetrvá, i když je neobratně a zmateně podaná. Chceme-li však myšlenku podat co nejvýstižněji a ušetřit tak čtenáři čas, musíme dodržet určité zásady, o kterých pojednáme dále.

\subsection*{Musíme vědět, komu to chceme říci}
Dalším důležitým předpokladem dobrého psaní je psát pro někoho. Píšeme-li si poznámky sami pro sebe, píšeme je jinak než výzkumnou zprávu, článek, diplomovou práci, knihu nebo dopis. Podle předpokládaného čtenáře se rozhodneme pro způsob psaní, rozsah informace a~míru detailů.

\subsection*{Musíme si dokonale promyslet obsah}
Musíme si dokonale promyslet a~sestavit obsah sdělení a~vytvořit pořadí, v~jakém chceme čtenáři své myšlenky prezentovat. 
Jakmile víme, co chceme říci a~komu, musíme si rozvrhnout látku. Ideální je takové rozvržení, které tvoří logicky přesný a~psychologicky stravitelný celek, ve kterém je pro všechno místo a~jehož jednotlivé části do sebe přesně zapadají. Jsou jasné všechny souvislosti a~je zřejmé, co kam patří.

Abychom tohoto cíle dosáhli, musíme pečlivě organizovat látku. Rozhodneme, co budou hlavní kapitoly, co podkapitoly a~jaké jsou mezi nimi vztahy. Diagramem takové organizace je graf, který je velmi podobný stromu, ale ne řetězci. Při organizaci látky je stejně důležitá otázka, co do osnovy zahrnout, jako otázka, co z~ní vypustit. Příliš mnoho podrobností může čtenáře právě tak odradit jako žádné detaily.

Výsledkem této etapy je osnova textu, kterou tvoří sled hlavních myšlenek a~mezi ně zařazené detaily.

\subsection*{Musíme psát strukturovaně} 
Musíme začít psát strukturovaně a~současně pracujeme na co nejsrozumitelnější formě, včetně dobrého slohu a~dokonalého značení. 
Máme-li tedy myšlenku, představu o~budoucím čtenáři, cíl a~osnovu textu, můžeme začít psát. Při psaní prvního konceptu se snažíme zaznamenat všechny své myšlenky a~názory vztahující se k~jednotlivým kapitolám a~podkapitolám. Každou myšlenku musíme vysvětlit, popsat a~prokázat. Hlavní myšlenku má vždy vyjadřovat hlavní věta a~nikoliv věta vedlejší.

I k~procesu psaní textu přistupujeme strukturovaně. Současně s~tím, jak si ujasňujeme strukturu písemné práce, vytváříme kostru textu, kterou postupně doplňujeme. Využíváme ty prostředky DTP\footnote{Desktop publishing (DTP) -- tvorba tištěného dokumentu na počítači.} programu, které podporují strukturovanou stavbu textu (předdefinované typy pro nadpisy a~bloky textu).

\subsection*{Nikdy to nebude naprosto dokonalé}
Když jsme už napsali vše, o~čem jsme přemýšleli, uděláme si den nebo dva dny volna a~pak si přečteme sami rukopis znovu. Uděláme ještě poslední úpravy a~skončíme. Jsme si vědomi toho, že vždy zůstane něco nedokončeno, vždy existuje lepší způsob, jak něco vysvětlit, ale každá etapa úprav musí být konečná.

\section{Komu se píše diplomka}
Tato podkapitola byla převzata z blogu prof. Herouta \cite{Herout}.

\bigskip
\noindent \bf Pište svou diplomku pro studenta, který má na Vaše dílo navázat. \rm
\bigskip

Představte si, že na Vaší práci bude dál pracovat student Franta, asi tak stejně chytrý jako Vy sami. Máte teď čtyři hodiny na to, abyste mu svou práci ukázali, zasvětili ho do~všeho, co je potřeba, a on pak bude pokračovat sám. Franta je studentem stejné školy jako Vy a~ví asi tolik, co průměrný student, není žádným super odborníkem na obor Vaší diplomky, ale rozhodně není hloupý a řešeného tématu se neštítí. Franta, tak jako Vy, se~o~tom, že bude po Vás pokračovat, zrovna dozvěděl, takže ještě neměl čas si něco k~tématu nastudovat.

Bude dobré začít tím, že se Franta dozví, co je cílem práce, proč se to celé dělá, co mají být výsledky.

Nikdo soudný by hodinu z vyměřeného času s Frantou nestrávil řečněním typu: \uv{\mbox{Internet} byl vytvořen americkou armádou v roce 1962, pak v roce 1991 v CERNu udělali www, a~nyní se používá v nejrůznějších oblastech lidské činnosti.} (to vše na šesti stranách s~mnoha odkazy a obrázky).
Franta obvykle nepotřebuje několikastránkové skriptum o detailech barevných modelů pro reprezentaci obrázků, historii a detaily Houghovy transformace, kompletní popis vrstev referenčního modelu ISO/OSI, ani řadu koláčových grafů o~zastoupení jednotlivých mobilních platforem na trhu za posledních deset let.
Franta potřebuje nasměrovat na~cenné zdroje, které Vám při řešení pomohly a chce letmý popis nástrojů a~algoritmů, které byly nutné pro řešení: \uv{Je potřeba nástroj XY, který slouží k tomuhle a~tamtomu, hlavně jeho modul PQ, který se používá tehdy a~tehdy. Nejlepší je k tomu tato dokumentace.}

Řekněte Frantovi hodně o tom, co se při řešení osvědčilo a co pomáhalo, ale upozorněte ho i na to, co nejdřív vypadalo jako dobrý nápad, ale pak se ukázalo jako zbytečné nebo nefunkční.

Dobře dávkujte úroveň detailu. Nějakou optimalizační fintu rozeberete ve zdrojovém kódu řádek po řádku, nějaký pomocný modul přejdete jedním odstavcem s popisem vstupů a výstupů a odkazem na použitou knihovnu.

Představte si průběh toho čtyřhodinového sezení s Frantou.
\begin{itemize}
  \item{O čem byste asi mluvili na začátku, kdy se Franta teprve začíná orientovat?}
  \item{Co jsou věci, které by rozhodně měly zaznít?}
  \item{Jaké obrázky byste v průběhu sezení malovali?}
  \item{Na co by se Franta vyptával, protože je to důležité a přitom to není samozřejmé?}
  \item{Na jaká omezení a nedodělanosti byste Frantu potřebovali upozornit, aby neupadl do~nějaké pasti?}
  \item{Jak vlastně Franta může pokračovat? Co jsou otevřené záležitosti, které by ještě stálo za to vyzkoušet a vylepšit?}
  \item{Co byste říkali úplně první (úvod) a úplně poslední (závěr) minutu sezení?}
\end{itemize}


\section{Struktura diplomové práce -- Pět kapitol}
Není-li dále uvedeno jinak, tato podkapitola byla převzata z blogu prof. Herouta \cite{Herout} (částečně inspirovaného knihou, kterou napsal Jean-Luc Lebrun \cite{Lebrun2011}) a z dokumentu na osobní stránce prof. Zemčíka~\cite{Zemcik}.
\bigskip

Diplomová práce je činnost, kterou student vyvíjí po dva semestry studia a pak o ní sepíše knížečku. Rozšířená terminologická chyba je, že se té knížečce, která je o činnosti sepsaná, říká diplomová práce. Ta knížečka je ve skutečnosti technická zpráva o provedené roční činnosti a diplomová práce je ta roční činnost.

Diplomantova roční činnost zahrnuje za prvé studium: \uv{Co už v oblasti mého zadání existuje? Jak to dělají jiní?} V rámci diplomky člověk dále nějaké věci vymyslí a navrhne: \uv{Zadaný problém lze řešit tak a nebo tak, já k němu přistoupím tímto způsobem, protože na~zvolené platformě je to nejefektivnější.} To, co řešitel navrhl, by měl po sobě ověřit tím, že to implementuje a vyhodnotí: \uv{Pro implementaci jsem zvolil ty a ty nástroje, celý systém rozvrhl do takových modulů. Výsledek je takhle rychlý, má takovou úspěšnost a~reakce uživatelů jsou takové a takové.}

Základní struktura diplomové práce podle prof. Herouta tedy je:
\begin{enumerate}
  \item{Úvod (1 strana)}
  \item{Co bylo třeba vystudovat (vč. zhodnocení současného stavu; 40\,\% rozsahu)}
  \item{Nové myšlenky, které tato práce přináší (30\,\%)}
  \item{Implementace a vyhodnocení (30\,\%)}
  \item{Závěr (1 strana)}
\end{enumerate}

Není chybou, když text má právě 5 takových kapitol, není ani chybou, když je některá z~nich rozdělena na dvě části -- o tom dále. Obvykle je velkou chybou, když tam některá část chybí, nebo má nápadně odchylný rozsah. Názvy kapitol nemusí kopírovat tuto strukturu. Samozřejmě, samotný obsah práce je nadřazen všem zásadám a pokud tedy bude dobrý důvod strukturu porušit, tak to udělejte.

Na této základní struktuře se řada vedoucích shoduje, byť různí vedoucí doporučují rozdílné názvy kapitol a např. zhodnocení současného stavu lze umístit nejen do 2., ale i~do~3. kapitoly jak to doporučuje prof. Zemčík:
\begin{enumerate}
  \item{Úvod (1--2 strany)}
  \item{Shrnutí dosavadního stavu (40--50\,\% celkového rozsahu)}
  \item{Zhodnocení současného stavu a návrh řešení (3--5 stran)}
  \item{Popis vlastní práce (cca 40\,\% celkového rozsahu)}
  \item{Závěr (max. 1 strana)}
\end{enumerate}

Názory vedoucích se dle zaměření práce liší i v rozsazích, jak je vidět např. v~doporučeních dr. Berana \cite{Beran}:
\begin{enumerate}
  \item{Úvod (1 stránka)}
  \item{Teorie (1/3 stran)}
  \item{Návrh řešení (1/3 stran)}
  \item{Realizace, experimenty a vyhodnocení (1/3 stran)}
  \item{Závěr (1 stránka)}
\end{enumerate}

U prakticky zaměřených prací, pro která jsou zásadní data a uživatelská rozhraní, lze využít doporučení od doc. Černockého \cite{Cernocky}:
\begin{enumerate}
  \item{Úvod (jednotky stran)}
  \item{Teoretická část (cca 10 stran)}
  \item{Data (jednotky stránek)}
  \item{Popis Vašeho algoritmu a jeho testování (cca 10 stran)}
  \item{Návrh a implementace (pár stran)}
  \item{Uživatelské rozhraní (pár stran)}
  \item{Testování (cca 10 stran)}
  \item{Závěr (jednotky stran)}
\end{enumerate}

\section{Diplomka -- komiksová edice}
Tato podkapitola byla převzata z blogu prof. Herouta \cite{Herout}.

Diplomka (bakalářka je taky diplomka) je poměrně komplexní dílo. Skládá se z velkého počtu písmenek. A ta písmenka nejsou jen tak za sebou, ale jsou uspořádána hierarchicky do~kapitol. Celé to musí mít nějakou logiku, nějaký sled -- čtenář se musí nejdřív dozvědět jedny věci, aby mu pak šlo přesvědčivě předat věci jiné. Musí tam být obrázky, tabulky, vzorce; zároveň si tyto ne-textové věci musí s okolním textem povídat a navzájem se doplňovat. Musí se zcela pokrýt oficiální zadání, a vše musí být doručeno k nějakému přesnému datu, vytištěné a svázané. Pokud, nad to všechno, má být diplomka dobrá, musí to všechno být uděláno dobře. Když se peče koláč z deseti surovin a jedna z nich je zkažená, celý koláč bude hnusný. Musí klapnout všechno.

\subsection*{Jak to všechno pohlídat? Odkud zaútočit?}
Když s kolegy píšeme článek (poslední dobou asi tak pořád), v hodně rané fázi připravíme něco, čemu říkáme \uv{Comics Edition}. Děláme to jednak proto, že já na tom trvám, a dvak proto, že nám to dosti pomáhá. Třeba to pomůže i Vám s diplomkou.

Nejprve si ujasněte odpovědi na následující otázky:
\begin{itemize}
  \item{Jak byste vystihli podstatu svého řešení ve třech až pěti krátkých větách?}
  \item{Jaké jsou silné stránky Vašeho řešení?}
  \item{Jaké jsou konkrétní argumenty, že to, co jste udělali, je dobré?}
  \item{Kdyby chtěl někdo být na Vaši práci zlý -- co by vytkl?}
  \item{Co byste mu odpověděli?}
  \item{Jaká klíčová slova by měl člověk zadat do vyhledávače, aby Vaše diplomka byla relevantní odpovědí?}
\end{itemize}

Pokud máte, můžeme jít na věc\ldots

\subsection*{Hned založit TEN dokument}
Občas vídám postup, že někdo píše \uv{předběžnou} verzi diplomky do nějakého poznámkovadla. Je to za prvé práce navíc a za druhé zbytečná. Je třeba hned založit dokument, ve~kterém svůj boj dokončíte, a ze kterého výsledek vytisknete.

\subsection*{Nadpisy kapitol}
Důležitou součástí komixového vydání, o něž se tu snažíme, jsou nadpisy kapitol. Vložte je do dokumentu. Vložte je tak, jak budou z hlediska formátování -- žádné provizorní seznamy: \uv{Já to pak předělám}. Chcete vidět, jak přesně to bude vypadat, jak se vyloupne celý automaticky sestavený obsah. Vložte je jak budou z hlediska jejich znění. Nadpis kapitoly říká, co v kapitole bude. Nadpisy kapitol tvoří kostru celého díla, již pak obalujete masem a kůží textu a obrázků.

Ze všech slov, která jsou v diplomce, jsou slova v nadpisech \textbf{ta nejdůležitější}. Věnujte jim opravdu mimořádnou pozornost.

\subsection*{Obrázky}
Obrázek vydá za tisíc slov. Prošel jsem 8 posledních článků, jež jsem spoluautoroval. Dohromady mají 80 stran, obsahují 87 obrázků a 17 tabulek, tj. 1,3 vizuálního sdělení na~stránku (včetně stránek obsahujících reference do literatury, úvodních stránek s abstrakty a vůbec všeho). Mnohé obrázky (tak půlka) se ve skutečnosti skládají z několika podobrázků, zvlášť odkazovaných. Těch jsem v řečených 8 článcích napočítal 221, tj. v průměru 3 vizuály na~stranu. Taková je moje představa o roli obrázků v odborném textu. Nemyslím si, že by mohla existovat vážně míněná diplomka, která by měla \uv{příliš mnoho obrázků}.

Již v rané verzi komixového vydání diplomky hleďte promyslet, kde se obrázky budou vyskytovat a jaké. Obrázky ještě nemusíte mít hotové. Zdaleka. Ještě nevíme, co přesně na~obrázku bude. Ještě nevíme, jaký pod ním bude popisek. Co víme je, že tu nějaký takový obrázek bude a že se bude skládat z více podobrázků a tak ho hned vložíme. Zabere to tak minutu (vektorovou podobu \uv{TODO Image} máme už dávno uloženou a je součástí této šablony) a hned se víc rýsuje, jak text bude vypadat. 
U některých obrázků už dokonce máme představu, jak budou vypadat -- konceptuálně. Nakreslíme na papír nebo na tabuli, vyfotíme mobilem, obrázek vložíme, aby držel místo tomu, jenž přijde po něm a bude nakreslený pořádně (vektorově v Inkscape\footnote{\url{https://inkscape.org}} nebo vygenerovaný Gnuplotem\footnote{\url{http://www.gnuplot.info/}}).

Jen tak na okraj: Obrázek vydá za tisíc slov. Hloupý obrázek vydá za tisíc hloupých slov. A když už jsme u těch obrázků: Kdo vkládá věci, které by měly být vektorové (schémata, grafy, nákresy, diagramy, prakticky všechno kromě fotek a~snímků obrazovek), jako rastrové obrázky, a kdo vkládá snímky obrazovek (a podobné věci, které mají být přesně) se ztrátovou kompresí (obvykle JPEG), nemůže očekávat pozitivní hodnocení práce.

\subsection*{Objem textu}
Zrovna tak jako vkládáme obrázky, které ještě nemáme, vkládáme i text, který ještě nemáme. V \LaTeX{}u je na to krásný příkaz \texttt{\textbackslash Blindtext}\footnote{Stručný tutorial: \url{https://texblog.org/2011/02/26/generating-dummy-textblindtext-with-latex-for-testing/]}}. Kdo ho (ke své škodě) nechce nebo neumí používat, použije \url{http://lipsum.com}. Pomůže pisateli tušit přibližný rozsah celého díla, hustotu obrázků v textu a další charakteristiky textu při pohledu shůry. Vytvořit si takový odhad trvá třeba 5 minut. Pro orientaci v rozdělaném textu je rozumné tento nijaký text vybarvit šedou barvou (inteligent si na to udělá příkaz, ať se pak barva snadno jednotně změní pro celý dokument). Zkušenost říká, že bez barev v tom člověk začne mít slušný nepořádek -- co už je hotové, co ne, na čem je potřeba pracovat. Je radno investovat pár minut do zprovoznění balíčku pro barvení textu. V rámci této šablony můžete použít příkaz \verb|\todo|, např. \todo{Toto je třeba dokončit}. 

Geneze každé kapitoly ať začíná tím, že obsahuje třeba 3-5 kusů TODO a nějaké to Lorem ipsum. Každé TODO se pak postupně transformuje na větší počet TODO menšího rozsahu, nebo na text, na obrázek, další podkapitoly, cokoliv. TODO střídavě přibývají a~ubývají, práce se přitom vždycky o kousek pohne.

\subsection*{Jak s tím celým pracovat}
Když na to člověk sedne a \LaTeX{} zrovna nemá špatný den, za hodinu je hotová diplomka (bakalářka je taky diplomka), o správném počtu stran, obsahující představu, co kde bude. Začíná se podobat výsledku, který má přijít až za nějakou dobu, po nějaké práci.

Dokument pak už v zásadě nenarůstá, ale transformuje se. Je velký rozdíl sednout si před prázdnou bílou obrazovku a \uv{psát diplomku} a vzít si jedno TODO a napsat místo něj odstavec. To první je těžké a někdy to prostě nejde. To druhé jde: má to svůj začátek a konec. Ví se, co se má udělat.

Pořád platí, že diplomka se nenapíše sama, ale jde to lépe a výsledek má spíše hlavu a~patu.

\section{Jak pojmenovat kapitoly}
Dobře publikovat neznamená jen posypávat co nejvíc papíru co nejvíce písmenky. Renomé vědce vzniká vlastně až tím, že jinému vědci je jeho práce natolik užitečná, že ho cituje ve~své práci. Proto je potřeba, aby svůj článek napsal dobře: nikdo nebude citovat práci, která je humpolácky napsaná, protože by se sám shodil. 
Humpolácky napsaný článek ovšem nikdo nebude citovat už z toho důvodu, že \bf ho nenajde\rm . Už dávno před nějakým internetovým SEO vědci používali při psaní různé \uv{fligny} tak, aby je další vědec, když dělá svoji rešerši, zařadil mezi svůj materiál, jejž přečte, z něhož si nadělá poznámky a který -- nakonec a~logicky -- ocituje ve svém díle.

Na světě je moře článků. Když vědec Tonda hledá materiál relevantní pro svou práci, zadá klíčová slova (kdysi papírově do knihovny, nyní elektronicky do příslušného vyhledávače) a vypadne mu hromada nálezů -- tj. názvů. První krok, aby článek někdo ocitoval, je mít dobrý název. Tak dobrý, aby Tondu zaujal a on si název rozkliknul ve snaze zjistit více. Název je \bf prvním filtrem\rm . 

Články, jež prošly prvním filtrem, si Tonda rozklikne. To znamená, že vidí abstrakt článku. Abstrakt je \bf druhým filtrem \rm a hodně na něm záleží. Je to jako dostat se do~druhého kola pohovoru k vysněné práci. 

Když tedy Tondu zaujme nadpis i abstrakt, stáhne si celé PDF článku a rychle ho proskroluje, aby si udělal představu: vytiskne ho, nebo okno s článkem zavře a bude se věnovat desítkám jiných? To je \bf třetí filtr \rm a je to jako být mezi pár posledními uchazeči o~práci snů. Na co se Tonda dívá ve třetím kole, během svého skrolování? Na vizuály, tj. na~obrázky, tabulky, vzorce, a právě na nadpisy kapitol. Projde Váš článek třetím filtrem? Bude s ním Tonda pracovat? Na obrázky se zaměříme jindy, tato podkapitola je o nadpisech.

S diplomkami to může být trochu jinak. Ne každý pisatel diplomky stojí o to, aby ji lidé četli. Tušíme, že existují tací, kteří si přejí spíše pravý opak. Pojďme ale v tomto návodu pracovat s hypotézou, že pisatel chce napsat dobrý text, který by mohl být lidstvu užitečný a stojí za čtení. Tj. za který je možné dostat rozumnou známku.

\subsection*{Klíčová slova -- půl úspěchu}
Jedna z nejlepších rad pro psaní článků (a obecně odborného textu), kterou jsem kdy slyšel, není úplně intuitivní a samozřejmá.
\bf Napište si klíčová slova, jež by člověk měl napsat do vyhledávače, aby mu vypadlo Vaše dílo jako relevantní odpověď. \rm 
Popusťte uzdu fantazii, klidně to vezměte ze široka. Přemýšlejte o aplikacích Vaší práce. O souvislostech. Sepište všechna klíčová slova, bude to na několik řádků. Klíčové slovo je i sousloví -- typicky dvou nebo tří slov. Vyberte z nich ta důležitá. K tomu je potřeba intuice a zkušenost. Kde ty vzít, nevím, ale vždycky se můžete s někým (např. vedoucím práce) poradit. Až budete psát svůj třicátý odborný text, půjde to celkem hladce.

\bf Všechna důležitá klíčová slova se musí objevit v nadpisu článku nebo v nadpisech kapitol. \rm 

\subsection*{Příliš obecný nadpis, příliš specifický nadpis}
Proč to s těmi klíčovými slovy? Protože nadpisy jsou směrovníky, ukazatele, které ukazují: \uv{Co hledáš, je tady!} Aby někdo o text stál, musí se v něm zorientovat. Potřebuje vědět, že text nabízí odpovědi na některé jeho otázky. Nadpisy mu v tom můžou pomoct, nebo ho přesvědčit o tom, že o text vlastně nestojí. Kdo si při stěhování napíše na všechny svoje krabice od banánů: \uv{RŮZNÉ VĚCI}, bude mít pravdivé a formálně správné popisky, ale nemusel nic psát. Obecný popisek k ničemu není.

Jednoslovný nebo dvouslovný nadpis kapitoly jsou obvykle podezřelé z toho, že jsou příliš obecné -- s výjimkou úvodu a závěru, kde názvy kapitol jsou kanonické (dané vyhláškou). Máte-li ve své práci jednoslovné nadpisy kromě dvou řečených, pravděpodobně je máte špatně. Název kapitoly, který by šel použít u jiné práce stejného oboru, třeba \uv{Implementace systému}, \uv{Základ zpracování obrazu}, \uv{Principy uživatelských rozhraní}, je podezřelý z~toho, že je příliš obecný. Lepší obvykle bude \uv{Implementace systému pro sledování pohybu much}, \uv{Algoritmy pro detekci objektů a sledování jejich trajektorií}, \uv{Principy uživatelských rozhraní jednoduchých webových systémů}.

Název kapitoly, který by šel použít na úplně rozdílných školách je prakticky vždycky špatně -- příliš obecný. Nadpis \uv{Teorie} by mohl být použit na lesnické univerzitě, v IT, na~právech, na vysoké škole mlékárenské a sýrárenské. Je špatně. Nadpis \uv{Studium existujících řešení} je špatně. \uv{Průzkum dostupných technologií} je špatně. Ještě jsem neviděl nadpis, který by byl příliš specifický a nemyslím, že by něco takového mohlo existovat. Může být špatně -- tedy nevystihovat, co se v kapitole nachází. Pokud ale vystihuje, nemůže být příliš specifický.

Nevybízím tím k nadpisům na pět řádků. Většina dobrých a specifických nadpisů bude na jediném řádku a budou mít v průměru kolem pěti slov. Tu a tam nadpis přeteče na~druhý řádek a bude k tomu dobrý důvod. Sestavit dobré nadpisy -- dosti specifické a přitom ne~příliš dlouhé -- není lehké, ale vyžaduje to přemýšlení. Jako každá lidská činnost, která má za něco stát.

\subsection*{Zkratky v nadpise}
Zkratky v nadpise nemají co dělat, pokud nejsou úplně super-notoricky známé (ČR, AIDS, IT).

Je možné v první kapitole vysvětlit nějaký pojem a uvést, že dále se bude vyskytovat ve~zkratce. Je možné tuto zkratku používat dále v textu druhé kapitoly bez dalšího vysvětlení. Není ale možné tuto zkratku používat v nadpise druhé kapitoly, protože nadpisy čte vědec Tonda už ve třetím filtru při rozhodování, jestli se vůbec do první kapitoly pustí. Pokud Tonda při rychlém skenování článku narazí na něco, co v něm vzbudí dojem, že text je nějaký divný, nesrozumitelný, vlastně neví, co se tam píše (to je případ zkratky v~nadpise), článek zavře a už ho neotevře.

Odkazy do literatury a odkazy na další objekty v článku (obrázky, nadpisy, \ldots) do~nadpisu nepatří a ještě jsem neviděl případ, kdy by tam byly potřeba (a to už jsem viděl hodně případů, kdy se tam vyskytovaly).

\section{Obecné rady zkušených vedoucích}

Tato podkapitola obsahuje vybrané rady od dalších zkušených vedoucích, pod jejichž vedením již byly obhájeny stovky prací a kteří věnovali nemalé úsilí sepsání svých rad a jejich vystavení na web. Pro kompletní texty neváhejte navštívit jejich stránky, které naleznete v~literatuře \cite{Beran}, \cite{BeranPDF}, \cite{Cernocky} a \cite{Zemcik}.

\subsection*{Obecné rady dle dr. Berana}
Tato podkapitola byla převzata ze stránek dr. Berana \cite{Beran}, \cite{BeranPDF}.

Jak psát/nepsát
\begin{itemize}
  \item{Kapitoly číslujte maximálně do druhé úrovně, nadpisy nižších úrovní volte jako nečíslované a neuvádějte je v obsahu, výsledná práce i obsah budou mnohem přehlednější.}
  \item{Logické členění -- každý celek -- celá práce, každá kapitola, každá podkapitola má: úvod, stať a závěr:
    \begin{itemize}
      \item{úvod -- sděluje, co je obsahem celku, co se dočteme, co se řeší, uvádí do kontextu,}
      \item{stať -- řeší kontext, problém, detailně specifikuje problém, způsob řešení, postup řešení, výsledek řešení,}
      \item{závěr -- rekapituluje úlohu, shrnuje dosažené výsledky a jejich podstatu, uzavírá celek.}
    \end{itemize}}
  \item{Jedná se o technický text, nedoporučuji příliš mnoho osobních pocitů a \uv{povzdechů}.}
  \item{Nepoužívejte množné číslo \uv{MY} jsme udělali, chtěli, apod.
    \begin{itemize}
      \item{používejte buď trpný rod, \uv{testy byly provedeny} namísto \uv{my jsme provedli testy} -- zejména v teoretické části, kdy jde o převzaté myšlenky,}
      \item{tam, kde chcete zdůraznit, že se jedná o Váš přínos, Vaši práci, Váš nápad, apod. tak použijte \uv{já} -- návrh řešení, experimenty, realizace,}
      \item{protože MY (já-Vy, Vy-čtenář, Vy-svět) jsme nic neudělali, VY jste udělal(a).}
      \item{(To, že někde používáte nápady vedoucího, neřešte, to se očekává, je to Vaše práce na~jeho téma.)}
    \end{itemize}}
  \item{U převzatých obrázků/myšlenek/tabulek použijte \bf citaci zdroje\rm }.
  \item{Každý \bf nadpis \rm (kapitoly či podkapitoly) by měl být následován odstavcem textu, který čtenáře informuje, co se dočte v následující části, a který čtenáře uvede do~následující problematiky.}
  \item{Nepodceňujte úvod a závěr.}
\end{itemize}


\subsection*{Obecné rady od doc. Černockého}

Tato podkapitola je převzata ze stránek doc. Jana Černockého \cite{Cernocky}.

\begin{enumerate}
  \item{Přečtěte si pár dobrých DP/BP a snažte se vstřebat, jak taková dobrá práce vypadá. Příklady Vám rád dá Váš vedoucí.}
  \item{\textbf{Čeština/Slovenština nebo English?}
    \begin{itemize}
      \item Pokud umíte slušně anglicky a Vaše práce má potenciál být čtena někde jinde než na FIT VUT (součást mezinárodního projektu, práce pro nadnárodní firmu, popis SW, který chcete dát na GooglePlay atd. atd.), velmi doporučuji psát anglicky. Můžete si 100x říkat, že to pak z češtiny přeložíte, ale už na to nikdy není čas. Navíc nemusíte psát háčky a čárky.
      \item  Pokud pracujete na BP/DP lokálního významu a víte, že Vaše angličtina za~moc nestojí, doporučuji naopak ušetřit sobě i vedoucímu a oponentovi utrpení a psát česky nebo slovensky. Rady pro psaní práce v angličtině a chyby, kterých se studenti často dopouštějí, najdete v~příloze~\ref{anglicky}. 
    \end{itemize}}
  \item{Nemějte strach z toho, že budete mít málo stran! Nerozepisujte se zbytečně, nekopírujte zbytečné věci z Wikipedie -- jen tím naštvete Vašeho vedoucího a oponenta. Pokud budete následovat doporučenou strukturu a budete psát o tom, co jste skutečně udělali, budete mít na závěr dost kvalitního materiálu. }
  \item{Pište průběžně -- nemusíte psát přímo text technické zprávy se vším možným formátováním, ale mějte aspoň nějaký soubor README, kam si budete značit, na čem děláte, průběžné výsledky, co jste četli, co jste použili, o čem to zhruba je, atd. Důrazně varuji před systémem \uv{hrnu práci, pak to celé napíšu} -- za půl roku už vůbec nebudete vědět, co jste dělali, a budete si na to (v lepším případě) horko těžko vzpomínat, v horším případě si to budete muset zopakovat. Průběžné psaní Vám také pomáhá strukturovat Vaší technickou zprávu.}
  \item{Používejte kontrolu pravopisu (spell-checker). Ušetřete vedoucího a oponenta opravování hloupých chyb (překlepů apod.). MS Word má kontrolu pravopisu dobrou, v~Linuxu je slušný ispall/aspell/hunspell (volá se např. z populárního textového editoru Emacs). Některé nástroje jsou nepoužitelné, např. ten v PSPadu propouští chyb jak ryb.}
	\item{Vedoucího průběžně seznamujte s aktuálním stavem práce a sdílejte s ním pracovní verze. Vedoucí by měl práci vidět zavčas! -- nejpozději přibližně 14 dní před odevzdáním, nejlépe průběžně po kapitolách (tabulky s výsledky/závěry třeba nemusí být ještě úplně hotové). Vedoucí Vám ji poškrtá, Vy se na něho naštvete, co si to dovoluje nad mým (přece tak krásným!) dílkem, za~půl dne vychladnete a zjistíte, že má vlastně pravdu, opravíte a další verze již bude mnohem kvalitnější. Pokud toto neuděláte a~odevzdáte verzi, kterou jste četl(a) jen Vy, případně nikdo, dostane \uv{plnou palbu} Vašich chyb oponent a podle toho bude vypadat jeho hodnocení.}
\end{enumerate}


\subsection*{Obecné rady od prof. Zemčíka}
Tato část textu je převzata z dokumentu na stránce prof. Zemčíka \cite{Zemcik}. Normálním písmem jsou uvedeny vybrané důležité obecně akceptované zásady, kurzivou jsou uvedena osobní doporučení \uv{navíc}. 

Obsah práce by měl mít na jednu stranu textu práce nejvýše jeden \uv{řádek}. Členění práce do podkapitol by mělo být (s výjimkou úvodu a závěru) rovnoměrné. Tato zásada má přednost dokonce před standardním \uv{desetinným řazením} práce\footnote{Desetinné třídění dle ČSN ISO 7144 a ČSN 01 6910 -- výtah viz \url{http://web.ftvs.cuni.cz/hendl/metodologie/doporuceniupravydizprace.pdf}}. Pokud se týká písma, je třeba typy písma \uv{neplýtvat}. Méně je v tomto případě lépe. Orientačně byste měli kromě základního písma, nadpisů, popisů obrázků a tabulek a rovnic, používat jen minimum fontů, například italiku nebo tučné písmo pro zvýraznění textu (a to nejlépe ani ne obojí najednou) a vhodné písmo s \uv{konstantním krokem} (pevnou šířkou) pro úryvky zdrojových textů. Kromě toho je třeba dodržet formální šablonu předepsanou pro práci, která je uvedena na~fakultním webu. 

\it Pokud se týká úvodu a závěru, velmi doporučuji je vůbec nečlenit do podkapitol a nechat je jako kompaktní bloky textu. Z výše uvedeného v podstatě taky vyplývá, že je vhodné kapitoly členit do podkapitol jen do \uv{1. úrovně}. Pokud byste cítili potřebu častějších nadpisů než v~průměru jednoho na 
stranu (zamyslete se ale v tom případě nad tím, jestli to je skutečně potřeba), dá se použít v textu \uv{pomocný} nadpis bez číslování a bez zařazení do obsahu. U~nadpisů je též rozumné se zamyslet nad tím, zda jsou pro čtenáře rozumným vodítkem o textu a pokud tomu tak není, je asi rozumné je přejmenovat. Je též dobré za každým nadpisem zařadit alespoň kousek textu (tedy za nadpisem kapitoly třeba \uv{2.~Stav práce} nepokračovat hned \uv{2.1 Počátek stavu}, ale zařadit kus textu a pak pokračovat. Není také dobré končit kusy textu obrázkem nebo rovnicí, ale je vhodné zakončovat nějakým shrnujícím textem. 

Typograficky podstatnými prvky jsou samozřejmě rovnice, obrázky, grafy, nadpisy apod. V~práci je ale jejich formát v podstatě \uv{tvrdě} upraven šablonou, takže se jimi zde nemá smysl zabývat příliš detailně. Přesto je třeba říci pár slov k jejich začlenění do práce. Především dbejte na to, aby byly 
tyto \uv{grafické} prvky dobře graficky odděleny od textu a aby byl výsledek \uv{graficky pěkný}. Výmluvy typu \uv{toto udělal \LaTeX}, \uv{toto udělal Word} neobstojí. U obrázků je vhodné dodržet jednotný styl jejich začleňování do textu a přitom dbát na to, aby obrázky byly buď centrovány nebo aby byly (pokud jsou užší než stránka) umisťovány u~vnější strany stránky (vnější vzhledem k budoucí vazbě, při jednostranném tisku tedy v~zásadě vpravo). 

\textit{Poznámka:} Pokud byste nebyli spokojení s tím, co je na začátku této kapitolky napsáno o obsahu a chtěli čtenáře \uv{lépe navést}, udělejte rejstřík -- to je jiný typografický útvar než obsah a tam můžete uvést hesel kolik se Vám jen zlíbí. Je to ale dost pracné a proto to příliš nedoporučuji. 
\rm

Při psaní práce je třeba používat zásadně spisovný jazyk a vyvarovat se hovorových výrazů, případně slangu (včetně slangu odborného). 

\it Při psaní práce velmi doporučuji se zaměřit zejména na následující body, v nichž se, bohužel, často chybuje: 
\begin{itemize}
  \item{Využívání 1. osoby jednotného čísla (\uv{já}) je potřeba velmi omezit. Využívání 1. osoby množného čísla, i když se často využívá v beletrii, je potřeba téměř eliminovat úplně. Existují následující výjimky:
    \begin{enumerate}
      \item{1. osobu jednotného čísla je možno využívat v úvodu a závěru práce jako prostředek sdělení \uv{osobního dojmu} (například \uv{Obvykle se to dělá takto\ldots , ale já jsem zvolil jiný postup\ldots}), lze toho využít i ve zhodnocení současného stavu, ale v žádném případě ne ve shrnutí současného stavu.}
      \item{1. Osobu množného čísla je možno využívat v případě, že označujete část práce, kterou jste nedělali sami, ale dělali jste ji v nějakém kolektivu. Vzhledem k tomu, že BP/DP by měla být v zásadě Vaším dílem, je tím potřeba velmi šetřit a z práce by mělo být jasné, že alespoň 90\,\% práce jste dělali sami (například \uv{Program jsem napsal sám, ale pro testování programu jsem požádal o pomoc spolužáky a~spolu jsme provedli experiment\ldots}). Předcházejte otázkám typu \uv{Vy jste tu práci nedělal sám?}}
      \item{Výjimkou z obou výše uvedených pravidel jsou matematické texty, kde se tradičně užívá 1. osoby množného čísla (například \uv{Mějme krychli se stranou A\ldots}), tam 1. osobu samozřejmě užívejte bez omezení.}
      \item{Další možnou výjimkou jsou řečnické otázky, pokud je tedy v práci používáte. Celkem doporučuji 1. osobu jednotného či množného čísla v práci použít maximálně cca 10$\times$ (mimo případ 3, tam se neomezujte), dolní limit není, ale zase tak 1-2$\times$ se to i hodí.}
    \end{enumerate}}
  \item{Anglické výrazy by se v práci, obecně vzato, neměly moc vyskytovat. Vzhledem k tomu, že v našem oboru se jim ale nevyhneme, doporučuji takový postup, při kterém při prvním výskytu nějakého odborného pojmu uvedete obě verze (českou a anglickou) s tím, že tu, kterou nadále nebudete využívat, uvedete do závorky třeba i s komentářem (například \uv{\ldots octree (oktalový strom, nadále bude využívána jen anglická forma, protože to je zvykem i mezi odborníky v oboru)\ldots}).}
  \item{Zkratky by se při prvním užití v textu měly vysvětlit. Alternativou (asi lepší, ale pracnější) je uvedení seznamu zkratek, kde se na jednom místě přehledně vysvětlí všechny zkratky.}
  \item{Budoucí/minulý/přítomný čas by měl být v práci pojat tak, že obecně vzato je práce popisem obecně platných skutečností (pak používejte přítomný čas) v kombinaci s popisem Vaší práce (ta už proběhla, 
tak používejte čas minulý). V plánech budoucí práce samozřejmě používejte čas budoucí. Nejvíc ze 
všeho ale v tomto případě použijte \uv{cit} a přizpůsobte jazyk situaci tak, aby se práce dobře četla.}
\end{itemize}
\rm



\chapter{Závěr}
\label{zaver}

V tomto textu bylo uvedeno, jak začít s tvorbou bakalářské či diplomové práce, napsat abstrakt, připravit základní strukturu práce a co uvést do jednotlivých kapitol. Při tom bylo vysvětleno, že bakalářská práce je také diplomová a je třeba k ní přistupovat stejně zodpovědným způsobem. Následně byla věnována pozornost bibliografickým citacím a formální stránce práce. V předposlední kapitole jsou uvedeny důležité informace k odevzdání v listinné i v elektronické podobě.

Je třeba zdůraznit, že diplomová práce je unikátním individuálním dílem, které vzniká pod vedením zkušeného odborníka. Ať už je v této šabloně uvedeno cokoliv, závazné jsou pouze oficiální pokyny na stránkách fakulty. Pro konkrétní diplomovou práci je potřeba vždy zvažovat, co je z výše uvedeného textu relevantní a co nikoliv a řídit se především pokyny vedoucího, který rozumí dané problematice a je tak schopen poskytnout ty nejlepší rady, co lze k práci dostat.

I přes velkou snahu nikdy není možné do šablony zahrnout všechny prvky, co budou při tvorbě práce potřeba, a zaručit, že po doplnění textu, obrázků, literatury apod. bude vše v~pořádku pro všechny možné diplomové práce. Bude-li někde delší text, než se předpokládalo, a zalomí-li se na dva řádky, bude-li v literatuře položka, se kterou nebyl otestován využitý styl, a v dalších případech může být výsledek neuspokojivý a může být potřeba do~šablony zasáhnout a chybu, která se projevuje třeba jen pro jednu práci ze sta, opravit. Výsledné PDF a následně i vytištěnou papírovou verzi je tedy vždy nutné pečlivě zkontrolovat a~nespoléhat se na to, že \uv{tohle přece generuje šablona, tak to musí být správně}. Najdete-li v šabloně nějaké chyby nebo budete-li mít návrhy na její vylepěšení, napište prosím na e-mail sablona@fit.vutbr.cz a pomozte nám s jejím vylepšováním. Veškeré připomínky a návrhy jsou vítány.

S kontrolou výsledku může výrazně pomoci vedoucí práce. Nelze však předpokládat, že vedoucí poslední noc před odevzdáním bude sedět v práci připraven na kontrolu desítek stran textu. Proto je nutné mít vše připravené v předstihu a konzultovat průběžně. Kritický pohled vedoucího pak umožní dosažení kvalitního výsledku a aktivita, kterou uvidí, přispěje k pozitivnímu hodnocení práce z jeho strany.

Na závěr bych jménem autorů této šablony popřál všem, kteří právě vytvářejí svoje diplomové práce nebo se k jejich tvorbě připravují, úspěšné dokončení a obhájení práce.






%===============================================================================
